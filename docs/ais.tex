\documentclass{article}
\newcommand{\Title}{Artificial Immune Systems}
\newcommand{\Author}{Lee Hall}
\newcommand{\Date}{Fri 07 Dec 2012 01:39:09 AM EST}
\usepackage{url}
\usepackage[T1]{fontenc}
\usepackage{amsfonts,amsmath,amssymb,amsthm}

\theoremstyle{plain} %default
\newtheorem{lem}{Lemma}
\newtheorem{thm}{Theorem}

%setup a case environment that resets at each new theorem, but doesn't label it
%with the theorem number
\theoremstyle{remark}
\newtheorem{case}{Case}[thm]
\renewcommand{\thecase}{\arabic{case}}

\usepackage[margin=1in]{geometry}
\InputIfFileExists{header.tex}{}{
	\title{\Title}
	\author{\Author}
	\date{\Date}
}
\begin{document}
\maketitle
\section{Abstract}
An Artificial Immune System (AIS) is a novel approach to guided
classification that combines traditional genetic algorithms with a model
of biological immune systems. Our approach is based on the work of
von Zuben and de Castro\cite{decastro}, which was in turn based on the
earlier work of D. Dasgupta\cite{dasgupta}. We attempted to apply these
techniques to hyperspectral image classification, by generalizing an
existing classifier that recognized vectors in $\mathbb{N}^3$ to work with
vectors in $\mathbb{R}^n$. This generalization revealed flaws in the underlying
codebase that gave inconclusive results on the original project goals,
but did provide some interesting information on the utility of various
parts of the AIS algorithm.

\section{Introduction}
One of the more common means of building a classifier is a Genetic
Algorithm (GA). A GA initially generates a random set of states. It then
applies a fitness heuristic to select the most fit states. It then
combines these states and applies random perturbations before repeating
the process in an attempt to converge on an ideal state\cite{norvig}.
The AIS system is a relative of the GA, also biologically inspired,
which uses simulated annealing to converge on a match for each element
of the training set. The matching set, known as the Antibody set, is
then produced by combining the matches with the highest fitness, and
then pruning elements that are too closely related to each other. 

\section{ImagePimp}
The starting point for our research was an existing application known as
ImagePimp, which was built to do basic image manipulation and testing.
There was an AINet module in this program, which would load a training
set from a text file and attempt to classify the existing image based on
that training set. After some work splitting the AINet code into an
independent class, an attempt was made to generalize the codebase. While
doing so, it became apparent that the code was extremely fragile and was
implementing an algorithm much closer to a random walk than the AINet
technique we desired.

\section{Benchmarking Harness}
When confronted by the AINet code that existed, it immediately became
clear that a testing harness must be implemented as quickly as possible
so that the performance of successive iterations could be measured and
regressions could be easily detected. As we had been provided with
several sets of classified data in text file format, scripts were
written to select a randomize portion of the classified data as a
training set, and to generate an unclassified copy of the original data
set. A database was built which could track these benchmarks by
recording the revision from the source control system as well as
relevant parameters, like the size of the Antibody set, and the number
of iterations allowed for convergence of the Antibody set. The
benchmarking utility used 4 data sets of classified and decoded picture
data: 2 ground\_training sets of 3 dimensional data, 1 iris set with 4
dimensions, and the wine set with 16 dimensions. Later a 2-d set was
generated based the SPIR problem\cite{decastro}. The initial results
were not auspicious: Table \ref{initial_res}.

\begin{table}
\begin{tabular}{cccccc}
% From revision 035d45d5f69815b8cb57d789c33004d256f2c5c7
       name       & iterations & round & Percent Correct &
Percent Stddev & trials \\
 ground\_training1 & 5 &  0.00 & 79.56 & 15.95 &     14\\
 ground\_training2 & 5 &  0.00 & 69.81 & 7.08  &     14\\
 iris             & 5 &  0.77 &  4.67 & 11.43 &     14\\
 wine             & 5 &  0.59 &  0.60 & 0.50  &     14\\
\end{tabular}
\caption{Initial Results}
\label{initial_res}
\end{table}
\cite{bishop}

\pagebreak
\bibliographystyle{ieeetr}
\bibliography{ais}
\end{document}


